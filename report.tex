\documentclass{report}

\usepackage{subfiles}

\title{Trajectory of habits and preferences in Parala}
\author{Rutwik Kharkar}

\begin{document}
\maketitle

\chapter*{Why Parala}

About a year ago, Prof. Levin put me in touch with Dr. Madhav Gadgil who, in addition to being one of India's premier ecologists, is also deeply involved in conservation work in many parts of the country. After exchanging a few emails, we both felt that it was important for me to personally observe the problems that conservation in India is facing. As a result, he put me in touch with one of the founders (Shantaram Bapu Pandere) of an NGO called `Bharatiya Lok va Paryavaran Vikas Sanstha' (Lokparyay for short).

This NGO works with tribal communities in a number of villages located close to Aurangabad. While the NGO is primarily concerned with obtaining land rights for tribal families, the founders of the NGO and their associates are actively working on many related issues including reforestation, education, research and marketing of non-timber forest products, women's rights, and conservation. With the founders' help, I have been visiting Parala, one of the villages they serve, in order to talk to the people there and to learn more about the social aspects of conservation in India.

I first visited Parala for two days in December last year. While there, I got a chance to talk to some of the tribal families. I stayed at a hostel that the NGO has helped build to aid in the education of tribal children, and so I was also able to interact with a few of the enrolled children. I also got to converse with the staff members that run the hostel and take care of the children. Additionally, I got to visit a biodiversity park that a few of the locals have started with the NGO's help. The biodiversity park is a restoration effort that, in addition to preserving biodiversity, also aims to preserve the knowledge and culture associated with this biodiversity. Although my visit was short, there were a number of things that were brought to my attention through the conversations I had with the villagers and with the founders of the NGO.

One of the things I learned was that many of the villages that the NGO serves are fairly remote and have had limited exposure to urban centres, but this is beginning to change. Roads and public transportation now connect these villages to neighboring towns and cities. Aurangabad, a rapidly developing small city, is no more than a few hours' bus ride away for residents of some of these villages. Weekly bazaars draw in people from a number of neighboring villages and serve as an important source of cultural exchange since the sellers and wares on display come from many different parts of the country.

Due to the limited contact, many of the tribal villagers have retained at least part of their traditional lifestyles. Especially those that now have land rights lead fairly autonomous lives based on subsistence farming, foraging, fishing, and animal husbandry. Some tribes also hunt. However, inclusion in the larger market economy of the country and to exposure to different cultures has led to some noticeable changes.

Although diets have remained fairly constant, other aspects such as clothing (both style and material) and language have changed. The biggest change that was evident to me was in the choice of crops on villagers' farms. Some part of each farm seems to be dedicated to growing cotton, which is primarily a cash crop. While the food crops themselves might not have changed much, the seeds for these crops do not come from native stock. Native varieties have been replaced by hybrid varieties that require large amounts of external inputs. Moreover, the seeds produces by these crops are not usually viable. Resultantly, the villagers are forced to buy seeds and chemical fertilizers and pesticides every year. Provided with adequate inputs, these new varieties produce very high yields. This enables the farmers to earn some money through their surplus produce, which in turn incentivizes them to keep farming these new varieties. Whether or not these new crops are truly profitable is an open question that I will discuss further in a later section.

Once I decided that I wanted to work on the question of how diets and dietary preferences affect biodiversity, I realized that this tribal community could be a good place to start. Particularly in order to investigate and identify the forces that influence diets and food choices in India, I thought that talking to the people here could be informative. The prevalence of a fairly traditional diet and lifestyle makes this a good baseline to compare other communities against. Any community that I will have access to will probably have undergone similar or higher levels of change. The green revolution has touched almost every part of India that is connected to the market system and native varieties of food and cash crops have been supplanted by hybrid varieties everywhere. The social and economic forces acting on the people here will likely be the same in other communities as well and will probably have manifested themselves to varying degrees.

In order to start my research, I spent a week in August 2017 in Parala and was able to interview a few different individuals and families. These conversations helped me learn a bit about the dietary choices of the people here and about some of the relevant economic and social forces. In conjunction with correspondence with a few other conservationists and scholars in India, these interviews have also provided me with an idea of what the future of Parala and its tribal inhabitants might look like. This report is written for the purposes of recording the ideas that arose out of my time in the village and to record and contextualize my predictions.

\tableofcontents

\chapter{Settings}
\section{Climate, terrain, and ecology}

The first thing that struck me as we were approaching the village was the near complete absence of vegetation. There are a few scattered shrubs growing in the rocky ground, but for the most part, the dirt road leading up to the village is surrounded by what appears to be desolate, barren land. I learned later that this entire area used to be heavily forested as recently as 40 years ago. The founders of Lokparyay, Bapu and Mangaltai Khinwasara, told me how they used to be scared of traversing this road when they first started working here because of the dense jungle. Some of the villagers themselves recalled childhood stories of how they would be stuck on a hilltop not daring to venture down because of the lack of visibility caused by dense vegetation. Only when they saw animals using some obscure trail could they descend.

The forest contained many different tree species. \textit{Saagwan} (Tectona grandis), \textit{palash} (Butea monosperma), mango (Mangifera indica), \textit{chincha} (Tamarin\-dus indica), \textit{amla} (Phyllanthus emblica), and a myriad of other woody species were commonly found here. There were a host of different \textit{ran bhajya} (forest vegetables) derived from herbs and vines, and shrubs that were also frequently encountered. All of the plants were known to the tribal people and were used for various different purposes including food, medicine, and construction material. The people depended on the forest for their sustenance, survival, and well being. Many of the plants also played important roles in religious ceremonies and cultural activities.

The reason for the disappearance of the forest is not yet a fully settled question in my mind. Apparently, foresters and the state government accuse the tribals of clear-cutting the forests for firewood and to practice their slash and burn form of agriculture. Conversely, Bapu, Mangaltai, and the tribals blame corrupt government officials for granting private businesses and prospectors unhindered access to the forest's extensive timber resources. These entities often employed tribals to cut down trees for them and payed them for the work. Fueled by rapidly growing cities and high demand for furniture in the big cities, they say, these profiteers irresponsibly abused the forest. Not being cognizant of the facts, I am unsure as to who is to blame, but given the presence of plantations on supposedly protected forest land and the injustices that the tribals are still subject to, I am more inclined to believe the latter story.

For the most part, the landscape is made up of gently sloping hills and vast tracts of open, rocky land. There were no large mammals that I could see, but tigers, a few different species of deer, wolves, and smaller cats like the jungle cat used to be quite common here. The villagers told me that the degradation of the forest was accompanied by the concurrent exodus of most animals, large and small. It is interesting to note however, that as soon as the biodiversity park was established, a number of smaller animals, particularly reptiles and insects, started returning promptly. A common Indian monitor (Varanus bengalensis) and a large constrictor have taken up residence within the park recently as well. Much to the consternation of the workers who painstakingly manage the park, some species of fatally poisonous snakes have also quickly returned.

Many bird species have also made a rapid local recovery. The hostel where I was living abuts the park and every morning I could hear peacocks, two different dove species, and a number of different passerines like orioles and purple sunbirds. Swallows, much more numerous in the dry season (August is in the latter half of the monsoon season), also made a few appearances.

The returning birds also brought new plant species with them. After the reappearance of the birds, new seedlings started to take root. Many different herbs, vines, and woody plants have now re-appeared within the park. The park has started expanding of its own accord. Given the rapid regeneration following re-introduction and care of a few native tree species, Lokparyay accuses the forestry department of gross mismanagement of the area's forest resources. They claim that following independence, if the department had simply left the forest alone, it would have taken care of itself. Instead, their meddling has led to the current sorry state of affairs.

Out of the park, even though exploitation of timber resources has now ceased, probably because there are no trees left, the local flora and fauna have had a much harder time reclaiming the land. Much of this can probably be attributed to the drought that has afflicted this region for the past 10 years. The monsoon season lasts from June until the beginning of October. From precipitation data available online, it looks as though Parala and neighboring villages have historically received around 600 mm of rainfall annually. Most of the precipitation occurs in July and September and there is negligible rainfall in the non-monsoon months. The rain also appears to be fairly uniformly distributed over the monsoon months with a few very heavy rains every year.

For the past 10 years however, much of the region has been afflicted by drought and very unpredictable rain. This year for example, Parala had gone 48 days without rain when I arrived in the middle of what should have been peak monsoon season. There is also much regional variation in rainfall. On the second to last day of my stay, Parala experienced about 15 cm of rain in the span of a few hours while villages only a km or two away only saw clouds and maybe a few drops of precipitation. A few of the older villagers I talked to spoke of times when they would have to be holed up in their houses for 2-3 days at a time because torrential downpours precluded any possibility of stepping outside. Far from being an aberration, this supposedly used to be a yearly occurrence but is now just a memory. I do not know if a strong scientific case can be made, but the villagers certainly seem to think that the loss of the forest has been responsible for the drought and their misfortune. If true, this would imply the presence of a feedback loop where lack of forest cover would result in lack of rainfall, which in turn would result in more trees dying because of the prolonged dry conditions. This would result in the forests further diminution.

Bapu thinks that climate change is primarily responsible for the reduced rainfall, but it is impossible for me to comment on the relative contributions of climate change and deforestation to the paucity of rain over the past decade.

Some seasonal streams, rivers, and lakes in this area also depend on the rain. Since many of them have run dry in the last decade, farming has become a very tenuous enterprise. In order to alleviate this problem, an irrigation project was initiated in 2002 and completed in 2003. An embankment was built across the biggest perennial water body in the area, the Manyad river. Many farms in Parala have benefited from this project. While a number of the tribal inhabitants in the village now have access to water for their crops because of this embankment, many are still dependent on wells and rainfall even though the embankment has the capacity to provide them with what they need. This is because of social inequities and discrimination that I will touch on in a later section.

\newpage

\subfile{people}

\end{document}