\documentclass[report.tex]{subfiles}

\begin{document}

\chapter{Summary and Plans for Future Investigations}

India today is in the middle of a wave of changes. Many of these changes are ones that seem to inevitably accompany economic development. The mass exodus of people from rural areas to cities, the rapid expansion of cities to accomodate growing populations, the homogenization of diverse peoples resulting from incorporation into a large, connected economy, changing patterns of land use, the increasing presence of industrially produced food in diets, and decreasing biodiversity are phenomena that almost all first world countries have experienced in their respective pasts.

These transitions interact with each other in ways that we do not understand well. In the past two decades, a fair amount of attention has been dedicated to questions of food security and on the environmental impacts of increasing demand for animal products in developing countries. However, greater consumption of livestock is part of a larger suite of dietary changes. In India, for example, wheat has come to replace millets in many households and has also replaced millets and pulses in many farms. In the North-Eastern states, potatoes have begun displacing other tubers such as yams and tapioca. Estimating the effects of such changes is difficult, in large part because of the paucity of data.

Moreover, while much of the research portrays effects as stemming from consumer choices (higher incomes = higher demand for animal products), it is not at all obvious to me that the story is so straightforward. Dietary changes usually seem to be well aligned with the profit maximizing food industry's interests. It might very well be the case that the food industry, in its capacity as a conduit between consumers and producers, influences both diets as well as land use. There is evidence showing the disproportional power of the food lobby in determining not just policy relevant to itself, but even national nutritional guidelines. There is little doubt that through a combination of advertising and aggressive lobbying, industry is able to significantly shape people's perceptions of food. Further, since food consumption is often a very visible behavior that can be used as a status symbol or for signaling, it is difficult to imagine dietary shifts as being independent of social forces.

While there is extensive research on determinants of individual food choice, there has been little insight into what causes large groups of people to change their diets. There is even less research causally linking changes in diet to changes in land use and thence to trends in species population. In order to begin addressing these gaps in our knowledge, I plan to undertake projects in India as well as in the US.

There are two primary aims for my research in India. The first is to identify agricultural land use changes that are occurring in parts of rural Maharashtra and to try to figure out why these changes are underway. The second is to begin documenting the effects of these changes on biodiversity. I will focus on plant and bird species (because those are the ones that I have a bit of a handle on). There is currently very little of such data available and I think it is vitally important to begin recording this information.

My time in Parala was a preliminary attempt at beginning this work. I had initially gone in thinking that I would be investigating dietary changes, but once I started talking to people there, I realized that since diets have remained unchanged for the most part, studying land use change would be more prudent. Moreover, other changes that are occurring might be precursors of changes in diets (see section \ref{sec:otherchanges} for these other changes). Many past Indian leaders have thought of rural India as a microcosm in which all the dynamics of the Indian macrocosm play out; so uncovering the reasons underlying changes here could illuminate causes of larger scale changes as well.

What I have seen so far in Parala is a long history of class and caste warfare, severe oppression of novices to the capitalist economy, and very rapidly changing biotic and abiotic features. I do not fully know why the trends I have observed (see Chapter \ref{chp:changes} for details on these changes) are occurring, but oppression by the upper classes seems to have forced the Bhils to adapt. Exposure to the market economy and the proliferation of media has also altered their expectations, aspirations, and perceptions. Improving economic conditions (for some of them) are allowing them to pursue these new aspirations.

The climate has been unfavorable for agriculture for about a decade now, although irrigation improvements are allowing farmers to hold on. Land cover has changed considerably. The once dense forest is now mostly gone with just a few broken remnants managing to survive in the more remote hills. Many large farms and plantations practicing mono-cropping have come up. The shifting agricultural style of the Bhils has disappeared, but recent developments have provided them with land deeds that enable them to sustain themselves. The Bhil farms, though small ($<$ 10 acres), are far more diverse than the bigger ones. They also use fewer synthetic inputs although this is also beginning to change because of the adoption of hybrid crop varieties.

Bapu has done some work documenting species on sustenance providing tribal farms as well as on larger, more business oriented farms. I am hoping to contribute to this work in December and to learn more about differences in biodiversity across the different kinds of land use. I am also planning on advancing this research by working with NGOs serving tribal populations in other parts of Maharashtra.

This research is probably not going to allow me to make any causal inferences about the relationship between diets and biodiversity. For this, I am going to rely on American history. One of the ways in which Americans' diets seem to have changed today is the large amount of farmed meat consumed. Until at least the early 1900s, game animals composed a more significant part of people's diets (according to Joel Greenberg, author of a few books on the natural history of the midwestern United States). The history of the game market might shed some light on this. Menus from Princeton's dining halls going back 100 years or so (available in the Mudd library) might also help me in identifying shifts in dietary trends. The FDA might also have some information on diets.

For changes in land use and for species' population trends, I am primarily going to focus on parts of the Illinois region. Illinois has witnessed a lot of change since the latter half of the nineteenth century. The invention of the self scouring steel plow promoted the establishment of monocultures of corn. By around 1900, most of the grassland in the region had been replaced by corn fields. More recently, soy has also gained a significant foothold. These and other changes could be gleaned from farming journals like the Illinois prairie farmer. Satellite imagery might also prove useful, but its utility might be limited if differentiating between corn and grass is technically difficult.

The Illinois Natural History Survey might be able to furnish some information on changes in species trends. An article I came across recently (Illinois Birds: A Century of Change) reports the results of bird surveys separated by 100 years. Currently active land retirement programs would also help in providing data for this study.

Discerning the interaction between diets and biodiversity will likely be difficult even if the individual trends are well documented. I think modelling will prove to be a critical part of this part of the project. Even for my investigations in India, using models to predict how different factors shape changes in land use (see section \ref{sec:otherchanges} for more explanation) could prove to be insightful.

Using American history to predict likely changes in India will of course be problematic. There are many differences between the two countries. They have very different histories, geographies, and societies. However, it might not be very far fetched to make some tentative remarks about how biodiversity in India might be affected by current diet trends on the basis of US history. While the two countries are different, the economic and social factors at play in these changes are similar. Many of the changes that India is experiencing today are ones that the US has already seen in the recent past. If there are lessons that can be learned from American history, it would be remiss to ignore them.

All in all, I think I have a better handle on the various projects that my thesis is going to entail than I did a few months ago. The next month is going to be important for setting up investigations in India that I will work on next summer. The upcoming semester is going to be dedicated to the US history part of my thesis. If everything goes according to plan, I should have an interesting story to tell in a couple of years time.

\end{document}