\documentclass[report.tex]{subfiles}

\begin{document}

\chapter{Summary and Plans for Future Investigations}

India today is in the middle of a wave of changes. Many of these changes are ones that seem to inevitably accompany economic development. The mass exodus of people from rural areas to cities, the rapid expansion of cities to accomodate growing populations, the homogenization of diverse peoples resulting from incorporation into a large, connected economy, changing patterns of land use, the increasing presence of industrially produced food in diets, and decreasing biodiversity are all phenomena that almost all first world countries have experienced in their respective pasts.

These transitions interact with each other in ways that we do not understand well. In the past two decades, a fair amount of attention has been paid on questions of food security and on the impacts of increasing demand for animal products on the environment in developing countries. However, greater consumption of livestock is part of a larger suite of dietary changes. In India, for example, wheat has come to replace millets in many households and has also replaced millets and pulses in many farms. Estimating the effects of such changes is difficult, in large part because of the paucity of data.

Moreover, while much of the research portrays the effects as stemming from consumer choices (higher incomes = higher demand for animal products), it is not at all obvious to me that the story is so straightforward. Dietary changes usually seem to be well aligned with the profit maximizing food industry's interests. It might very well be the case that the food industry, in its capacity as a conduit between consumers and producers, influences both diets as well as land use. There is evidence showing the disproportional power of the food lobby in determining not just policy relevant to itself, but even national nutritional guidelines. There is little doubt that through a combination of advertising and aggressive lobbying, industry is able to significantly shape people's perceptions of food. Further, since food consumption is often a very visible behavior that can be used as a status symbol or for signaling, it is difficult to imagine dietary shifts as being independent of social forces.

While there is extensive research on determinants of individual food choice, there has been little insight into what causes an entire population to change its diet. There is even less research causally linking changes in diet to changes in land use and thence to biodiversity. In order to begin addressing these gaps in our knowledge, I plan to undertake further projects in India as well as in the US.

There are two primary aims for my research in India. The first is to identify agricultural land use changes that are occurring in some parts of rural Maharashtra and to try and figure out why these changes are underway. The second is to begin documenting the effects of these changes on biodiversity. I will focus on plant and bird species (because those are the ones that I have a bit of a handle on currently). There is currently very little of such data available and I think it is vitally important to begin recording this information.

My time in Parala was a preliminary attempt at beginning this work. I had initially gone in thinking that I would be investigating dietary changes, but once I started talking to people there, I realized that since diets have remained unchanged for the most part, studying land use change would be more prudent. However, other changes that are occurring there do shed some light on why diets might be changing (see section \ref{sec:otherchanges}). Many past Indian leaders have thought of rural India as a microcosm in which all the dynamics of the Indian macrocosm play out; so uncovering the reasons underlying changes here could illuminate causes of larger scale changes as well.

What I have seen so far in Parala is a long history of class and caste warfare, severe oppression of novices to the capitalist economy, and very rapidly changing biotic and abiotic features. I do not yet know why the trends I have observed (see Chapter \ref{chp:changes}) are occurring, but oppression by the upper classes seems to have forced the Bhils to adapt. Exposure to the market economy and the proliferation of media has also altered their expectations, aspirations, and perceptions. Improving economic conditions (for some of them) are allowing them to pursue these new aspirations.

The climate has been unfavorable for agriculture for about a decade now, although irrigation improvements are allowing farmers to hold on. Land cover has changed considerably. The once dense forest is now mostly gone with just a few broken remnants managing to survive in the more remote hills. Many large farms and plantations practicing mono-cropping have come up. The shifting agricultural style of the Bhils has disappeared, but recent developments have provided them with land deeds that enable them to sustain themselves by producing their own food. The Bhil farms, though small (<10 acres), are far more diverse than the bigger ones. They also use fewer synthetic inputs although they have begun growing hybrid varieties of crops that require them to provide chemical fertilizers and pesticides.

\end{document}