\documentclass[report.tex]{subfiles}

\begin{document}

\section{The people}

Lokparyay aims to help many underrepresented and historically subjugated classes of people. Included among these are members of the Bhil, Thakar, and Paradhi tribes, scheduled castes, and a few different nomadic tribes. The people I interviewed during my visit this past summer belong to the Bhil tribe. Bhils comprise the majority of the underrepresented in Parala.

There does not seem to be much known about the Bhils. This seems strange given that they are the third largest scheduled tribe in India. Spatially isolated populations of Bhils are present in many different regions of western and central India. In total, according to a 2011 census, their population adds up to about 16 million individuals. The spatial isolation of different groups has led to two hypotheses about Bhil origins:
\begin{enumerate}
\item Either that the term Bhil has been used to describe a number of unrelated forest dwelling tribes
\item Or that the Bhils stem from a single tribe and their current spatial segregation is a result of environmental and social pressures.
\end{enumerate}

There is not much evidence to corroborate or disprove either hypothesis yet. Although Bhils from different regions recognize each other as members of a single group, many differences exist. Tribe members from different regions do not seem to conform to any particular body type. Further, although all the tribes now speak languages that are composites of various Indo-Aryan languages, there are local differences. Also, there is no evidence in their languages of elements pre-dating the Aryan invasion of India even though it is commonly accepted that Bhils were already well settled in India when the Aryans arrived. Economically and culturally as well, there is tremendous diversity across the few groups that have been studied.

While there is some literature on the Bhils of Rajasthan, there is little to be found about the Bhils in the hills surrounding Aurangabad. The only reference I have been able to find so far is in an appendix in the 1941 census of India. There are some references to the Bhils in Indian mythology, but like most Indian history today, their history essentially begins from colonial times.

The Bhils were most likely a forest dwelling people that practiced a form of shifting agriculture. There are many references to them in the Ramayana and the Mahabharata. In fact, based on accounts of his life, the author of the Ramayana, Valmiki, himself was a Bhil. There are many different mythological stories about the origin of the Bhils, including one that involves a great flood. The flood was survived only by a \textit{dhobi} (clothes launderer) and his sister. Their first son went into the forest and the Bhils are believed to be his descendants. Other origin related stories include curses, blessings, magic, greed, love, and treachery.

In more modern times, there are some references to them as early as about 600 A.D., but then there is a large gap until the arrival of British writers and historians. Bhils residing in areas near the one where I visited have apparently had run-ins with many different invaders. They were initially disposessed of some of their lands by the Rajputs probably around the 6th or 7th centuries. This was followed soon after by the arrival of the Marathas. While the Rajputs treated the Bhils quite generously, the Marathas were much less tolerant. There are stories of Bhils being flogged and hanged without question, of being thrown over cliff-sides by the hundreds, and of their women and children being mutilated and killed by the Marathas. It is probably during this time that their range contracted to the hills of this region from where they retaliated by raiding the villages in the plains. Travelers passing through the hills were always in danger of being attacked and robbed.

The Bhils have long been associated with dacoity and great cruelty. The areas they inhabited were considered extremely unsafe unless one traveled as a member of a large party. Especially during colonial times, the Bhils were responsible for numerous train and stage-coach robberies. The people I talked to, however, provided me with a very different perspective.

The Bhils consider their actions to be completely justified as the robberies were their way of fighting for freedom and helping out the poor. They consider themselves to be modern day Robin Hoods who stood up to the ruling classes and served those who were oppressed and had no power. They revere the leaders who often led these heists. There are tales told about the leaders' resourcefulness and intelligence and how it would have been impossible for the British to catch them if they had not been betrayed. Even today, there is a lot of resentment amongst the people against those who mistreated and villainized them.

Although India has now been independent for 70 years, the lot of the Bhil in these parts has not improved much. The stories that people told me about their childhoods and most of their adult lives were ones of severe want and hardship. While the ruling class might have changed, the Bhil's position in modern society is pretty much the same as it was pre-Independence. This is not surprising given that the mode of development and progress that independent India has adopted is strongly anti-rural. The Bhils in their current state as only incidental contributors to the Indian economy are easy to overlook and exploit. Recent improvements that have come about in their standards of living can almost exclusively be attributed to the efforts and activism of Bapu and Mangaltai, the founders of Lokparyay.

\newpage

\section{Lokparyay}

Although only recognized as a registered NGO in 2002, the founders of Lok va Paryavaran Vikas Sanstha have been working with Bhils, Paradhis, Thakars, and other disadvantaged communities in the area since the 1970s. The biggest hurdle for them has been in acquiring land rights for members of the communities they serve. When asked about their lives before they were granted formal land rights, my interviewees told me about how forest rangers were free to seize whatever land they wanted to, whether it was cultivated or not; how the more prosperous Maratha farmers would often graze cows on their crops; how their produce and grains were often stolen; and just how difficult it was to do any sort of agriculture. In order to get by, they had to work for their food and clothes (very rarely would they be paid in money) at village dwellers' houses.

\end{document}