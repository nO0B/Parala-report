\documentclass[report.tex]{subfiles}

\begin{document}

\section{The people}

Lokparyay aims to help many underrepresented and historically subjugated classes of people. Included among these are members of the Bhil, Thakar, and Paradhi tribes, scheduled castes, and a few different nomadic tribes. The people I interviewed during my visit belong to the Bhil tribe. Bhils comprise the majority of the underrepresented in Parala.

There does not seem to be much known about the Bhils. This seems strange given that they are the third largest scheduled tribe in India. Spatially isolated populations of Bhils are present in many different regions of western and central India. In total, according to a 2011 census, their population adds up to about 16 million individuals. Their spatial isolation has led to two hypotheses as to their origins:
\begin{enumerate}
\item Either that the term Bhil has been used to describe a number of unrelated forest dwelling tribes
\item Or that the Bhils stem from a single tribe and their current spatial segregation is a result of environmental and social pressures.
\end{enumerate}

There is not much evidence to corroborate or disprove either hypothesis yet. Although Bhils from different regions recognize each other as members of a single group, many differences exist. Tribe members from different regions do not seem to conform to any particular body type. Further, although all the tribes now speak languages that are composites of various Indo-Aryan languages, there are local differences. Moreover, there is no evidence in their languages of elements pre-dating the Aryan invasion of India even though it is commonly accepted that Bhils were already well settled in India when the Aryans arrived. Economically and culturally as well, there is tremendous diversity across the groups that have been studied.

While there is some literature on the Bhils of Rajasthan, there is little to be found about the Bhils in the hills surrounding Aurangabad. The only reference I have been able to find so far is in an appendix in the 1941 census of India.

\end{document}