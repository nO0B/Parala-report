\documentclass[report.tex]{subfiles}

\begin{document}

\section{The people}

Lokparyay aims to help many underrepresented and historically subjugated classes of people. Included among these are members of the Bhil, Thakar, and Paradhi tribes, scheduled castes, and a few different nomadic tribes. The people I interviewed during my visit this past summer belong to the Bhil tribe. Bhils comprise the majority of the underrepresented in Parala.

\subsection{Living conditions}

Many of my interviews were conducted in or just outside participants' homes. The houses I visited consisted of 1-3 rooms. In the houses with multiple rooms, one seems to be reserved for entertaining visitors. The kitchen, consisting of a \textit{chulha} (earthenware stove), a few utensils, and some kitchen supplies, is in another room that is also used as a bedroom and storage room. In the houses with one room, the kitchen is in a corner of the room.

While talking to me, the people either squatted or sat cross-legged on the floor, and I was either given a \textit{khatiya} (rope bed) or blanket to sit on. On a few occasions, women squatted on a smooth, gently sloping, low rock. A thin curtain separates one room from the other.

The Bhils that I visited either lived in the village of Parala itself or in the hills surrounding the village. In the village, the Bhil houses are concentrated in a small group along the outskirts. Apparently, every caste lives in a different part of the village. There is a well within each caste-based settlement and no individual from a lower caste is allowed to drink from the well belonging to a higher caste. No Bhil, for example, is allowed to drink from or even touch the water in a Maratha well. Farms are located outside the village boundaries.

Sanitation and facilities differ markedly between the Bhil occupied areas and those occupied by higher castes. While most of the village is serviced by public restrooms and a sewer system, the area occupied by the Bhils lacks any of this. The roads in most parts of the village are also wider and better maintained. The roads in the Bhil settlement are usually no more than dirt tracks. Even the Bhil farms are often worse off. Farms owned by the more prosperous and higher caste landowners draw all the water they need from the Manyad dam, but lack of resources and political clout often prevent the tribal people from making use of this amenity.

In the hills outside the village, houses are much more sparsely distributed. Residences are situated on hilltops from which the family's farm is usually visible. Neighbors seem to be separated by distances of at least half a kilometer. While all Bhils own domestic animals, they are much more evident here than in the village. Many goats and hens can be seen milling around. Cattle are also commonly owned, but the cowsheds are located elsewhere. During the day, the cattle and older goats are taken to graze while the young animals and fowl stay close to home. In the evening, the goats are brought back to the houses and the cows to their sheds.

The residences are either semi-permanent or permanent structures. I do not know anything about the materials that they are constructed from or how they are constructed, but I was told that they are a significant step up from the lean tos that were their childhood homes when they were a more nomadic people. While the walls of the permanent homes are solid, thick plastic tarps constitute the outer walls of the less permanent ones.

A family living together consists of the patriarch and his wife and his unmarried sons and daughters. Three seems to be the minimum number of children per married couple. Married sons and their families usually settle down near their parents and help with the farm work. However, sons may sometimes to choose to move farther away, particularly if his wife and parents do not get along. Daughters move to their husbands' villages and visit their parents only on special occasions like religious festivals and births and deaths in the house.

\subsection{Food}

As mentioned in the previous section, most of the food a Bhil family consumes comes from their own farm. The crops grown are drought resistant, dry region crops. Many legumes and a few cereals are well suited to the conditions here. Chief among beans and lentils are \textit{math} (Vigna aconitifolia), \textit{kulith} (Macrotyloma uniflorum), \textit{moog} (Vigna radiata), \textit{bahimung}, \textit{toor} (Cajanus cajan), \textit{waal} (Lablab purpureus (?)), and \textit{harbhare} (Cicer arietinum). The main cereal is \textit{bajri} (Pennisetum glaucum). Wheat is a less commonly grown cereal. Corn is grown on the larger farms, but I did not see any on the few Bhil farms I visited. Radishes and onions are grown as well.

A significant part of their food is foraged. If they see a plant in the wild that they know and like, they will take it home with them to eat. They often let what would usually be considered to be weeds grow in their fields if they know that it is edible. After one of my interviews, the hosts took me to their farm to taste the leaves of an \textit{ambadi} (Hibiscus cannabinus) plant that had inadvertently established itself there. It was fairly sour, and to me, delicious.

Many different parts of wild plants are used. Some recipes are made from flowers, others from leaves, and others from seeds and fruit. The plants and dishes that I commonly heard about while talking to the people were: \textit{koyad, troth, gethyacha phool, phangyachi bhaji, tandudka, kundra, vatla, patri, shevgyachi bhaji}, and \textit{saratechi bhaji}. I had never heard of any of these growing up, and neither had my other city dwelling relatives and friends that I mentioned them to. The fruit from \textit{amla, chincha, sitaphal, mango} and many other trees and shrubs are also harvested and consumed in various forms. A number of plants also have medicinal uses.

While some leafy vegetables like spinach and fenugreek are also sometimes grown, these and some other vegetables usually come from the weekly vegetable market. Rice, cooking oil, salt, and spices are also purchased from the market. Purchased food accounts for a small, but important part of their diet.

Crabs, fish, chicken, and milk (both cow and goat) are the primary sources of animal protein. Crabs and fish are caught in the nearby water-bodies, while poultry, cattle, and goats are domestic animals. Mutton consumption seems to be reserved primarily for special occasions. None of my interviewees hunt any more, although there is a lot of extant knowledge about how to trap different animals, when they are abundant, and what the animal meat is useful for. For example, the meat of \textit{pavat}, a species of dove, is supposed to be eaten if a fish bone is stuck in one's palette or digestive tract. It is believed to soften and melt the bone.

\subsection{Livelihoods}

A Bhil family with privately owned land is fairly autonomous. The size of a farm varies between 3 and 10 acres. In a good year, they can get 2-3 harvests from their farms, which, supplemented with foraged food, milk and meat products, and some food purchased from the market, is enough to meet their dietary need. The main sources of expenditure and dependence on external markets are clothes, farming and household equipment, and some food products that they cannot produce on their farms. Some money is also spent on occasional doctors' visits. Cash crops such as cotton are their primary source of income. Surplus produce and a few other jobs supplements this income.

The much larger farms in the neighborhood employ the Bhils as laborers. Men are paid more than women although it seems as though women work longer hours and do more intricate work than the men do. This used to be one of the primary sources of their livelihood, although with the acquisition of private land, they are not as dependent on this source of income anymore. Some of the tribals are also employed by the forest department in various capacities. One of the people I talked to had been hired to plant trees during one of the many reforestation drives and is now responsible for taking care of those and other trees. Many locals are hired temporarily during planting projects.

Neighbors help each other out in difficult times, but the recent prolonged drought has adversely affected many of the tribals. Unable to procure the required sustenance from their land, they are forced to seek opportunities elsewhere. A few of the most affected people work in sugarcane plantations in Gujarat and other parts of the country. They spend 6-9 months of the year at the plantations and return in the summer months to help their families in the farms. The work at the plantations is extremely strenuous and the working conditions are appalling. They are required to work at whatever hour their employers tell them to, be it in the darkness of the middle of the night or under the scorching midday sun. Since this seems to be the safest and surest source of income, the Bhils in dire need keep enduring it.

Most parents who work on the plantations take their children with them. Since these children spend so much time traveling and away from home, education is impossible for them. With the opening of the hostel in Parala however, some of the more fortunate children can stay back and attend school year round.

\subsection{Education}

Education among the Bhils used to be a luxury, but is now increasingly seen as a necessity for survival in what they perceive to be a world of relentless competition. Only one of the people I talked to had completed his secondary school education, although it had taken him a long time because of trying childhood circumstances. The others had been in and out of school and, after a few years, had given it up completely. Most of them were unable to get through primary school since they had to start working as farmhands and household servants at a very young age. During our conversations, many of my interviewees expressed great regret at not being educated. They talked about the many ways in which they are disadvantaged simply because of not being able to read and write. Many children today seem to have a much better chance of availing of education because of the greater security afforded by the recently acquired land rights and the hostel.

Locally available opportunities for education are not very good however. Government funded schools in India severely lack funding and infrastructure. The lack of funding translates into state run schools not being able to attract talented teachers. The teachers that are employed are not well trained and are usually not qualified to be teaching at all. On the few occasions when the teachers do show up to work, they are usually at least an hour late and they customarily leave early. Apparently, catching the teachers napping during classes is not uncommon. I asked Bapu why the children even need to go to school if they do not learn much there and in fact learn much more from their caretakers in the hostel. He told me that in order to obtain hall passes for board examinations, the children need to have a certain level of attendance.

Increasingly, as the Bhils become aware of opportunities elsewhere, they are sending their children to more distant privately funded English boarding schools where the educational prospects are far better.

\subsection{Personality}

Most of my interviewees, both men and women, were easy to talk to and were forthcoming with their answers to my questions. After the first few minutes, the conversations would become quite free flowing and very informative. To me, they seemed like simple, honest, and trusting people. Almost all of them were fluent Marathi speakers (much more so than me) and were good story tellers. There seemed to be no hesitation on their part in telling me about all parts of their lives, even the difficult and distressing ones.

They also seem to be a very passionate people. Often when they were telling me about the injustices that they have been subjected to, their voices would rise and their manner of speaking would change noticeably. They certainly do not seem to think very highly of foresters and the Indian government. At the same time, there are people that they are extremely devoted to. Almost everyone I talked to spoke reverentially about Bapu and Mangaltai and expressed gratitude for all that they have done. A few others that have worked amongst and for them also seem to have gained their respect.

\subsection{History}

There does not seem to be much known about the Bhils. This seems strange given that they are the third largest scheduled tribe in India. Spatially isolated populations of Bhils are present in many different regions of western and central India. In total, according to a 2011 census, their population adds up to about 16 million individuals. The spatial isolation of different groups has led to two hypotheses about Bhil origins:
\begin{enumerate}
\item Either that the term Bhil has been used to describe a number of unrelated forest dwelling tribes
\item Or that the Bhils stem from a single tribe and their current spatial segregation is a result of environmental and social pressures.
\end{enumerate}

There is not much evidence to corroborate or disprove either hypothesis yet. Although Bhils from different regions recognize each other as members of a single group, many differences exist. Tribe members from different regions do not seem to conform to any particular body type. Further, although all the tribes now speak languages that are composites of various Indo-Aryan languages, there are local differences. Also, there is no evidence in their languages of elements pre-dating the Aryan invasion of India even though it is commonly accepted that Bhils were already well settled in India when the Aryans arrived. Economically and culturally as well, there is tremendous diversity across the few groups that have been studied.

While there is some literature on the Bhils of Rajasthan, there is little to be found about the Bhils in the hills surrounding Aurangabad. The only reference I have been able to find so far is in an appendix in the 1941 census of India. There are some references to the Bhils in Indian mythology, but like most Indian history today, their history essentially begins from colonial times.

The Bhils were most likely a forest dwelling people that practiced a form of shifting agriculture. There are many references to them in the Ramayana and the Mahabharata. In fact, based on accounts of his life, the author of the Ramayana, Valmiki, himself was a Bhil. There are many different mythological stories about the origin of the Bhils, including one that involves a great flood. The flood was survived only by a \textit{dhobi} (clothes launderer) and his sister. Their first son went into the forest and the Bhils are believed to be his descendants. Other origin related stories include curses, blessings, magic, greed, love, and treachery.

In more modern times, there are some references to them as early as about 600 A.D., but then there is a large gap until the arrival of British writers and historians. Bhils residing in areas near the one where I visited have apparently had run-ins with many different invaders. They were initially disposessed of some of their lands by the Rajputs probably around the 6th or 7th centuries. This was followed soon after by the arrival of the Marathas. While the Rajputs treated the Bhils quite generously, the Marathas were much less tolerant. There are stories of Bhils being flogged and hanged without question, of being thrown over cliff-sides by the hundreds, and of their women and children being mutilated and killed by the Marathas. It is probably during this time that their range contracted to the hills of this region from where they retaliated by raiding the villages in the plains. Travelers passing through the hills were always in danger of being attacked and robbed.

The Bhils have long been associated with dacoity and great cruelty. The areas they inhabited were considered extremely unsafe unless one traveled as a member of a large party. Especially during colonial times, the Bhils were responsible for numerous train and stage-coach robberies. The people I talked to, however, provided me with a very different perspective.

The Bhils consider their actions to be completely justified as the robberies were their way of fighting for freedom and helping out the poor. They consider themselves to be modern day Robin Hoods who stood up to the ruling classes and served those who were oppressed and had no power. They revere the leaders who often led these heists. There are tales told about the leaders' resourcefulness and intelligence and how it would have been impossible for the British to catch them if they had not been betrayed. Even today, there is a lot of resentment amongst the people against those who mistreated and villainized them.

Although India has now been independent for 70 years, the lot of the Bhil in these parts has not improved much. The stories that people told me about their childhoods and most of their adult lives were ones of severe want and hardship. While the ruling class might have changed, the Bhil's position in modern society is pretty much the same as it was pre-Independence. This is not surprising given that the mode of development and progress that independent India has adopted is strongly anti-rural. The Bhils in their current state as only incidental contributors to the Indian economy are easy to overlook and exploit. Recent improvements that have come about in their standards of living can almost exclusively be attributed to the efforts and activism of Bapu and Mangaltai, the founders of Lokparyay.

\newpage

\section{Lokparyay}

Although only recognized as a registered NGO in 2002, the founders of Lok va Paryavaran Vikas Sanstha have been working with Bhils, Paradhis, Thakars, and other disadvantaged communities in the area since the 1970s. The biggest hurdle for them has been in acquiring land rights for members of the communities they serve. When asked about their lives before they were granted formal land rights, my interviewees told me about how forest rangers were free to seize whatever land they wanted to, whether it was cultivated or not; how the more prosperous Maratha farmers would often graze their livestock on the tribal people's crops; how their produce and grains were often stolen; and just how difficult it was to practice any form of agriculture. In order to get by, they had to work for their food and clothes (very rarely would they be paid in money) at the more prosperous village dwellers' houses and farms. It appears that all of the Bhils of this region have seen days of severe hardship and want.

\end{document}