\documentclass[report.tex]{subfiles}

\begin{document}

\chapter{Changes}\label{chp:changes}

The Bhils have witnessed the rise and fall of numerous cultures and a few illustrious civilizations over the many centuries of their existence (subsection \ref{subsec:history} in the previous chapter provides a very brief history of the Bhil people). Some experiences with other cultures have been affable, while others have been markedly hostile. Bhil traditions have likely changed repeatedly in response to these encounters (probably more in response to the hostile ones).

Their most recent major run-in has been with a culture founded on capitalism and individual rights. Exposure to this culture probably began in the British colonial era with the privatization of the forests that used to be their homes. Exposure has continued post independence as free India has continued along a path of development that it was introduced to by the British. This path is decidedly anti-rural and has forced the Bhils to adapt by changing many aspects of their lives. This chapter deals with what I think some of these changes are. Similar trends are being observed all over India but to my knowledge, not much has been done to understand the social and economic factors driving them. Therefore, in this chapter I will also attempt to identify the forces that I think are important. Although people's diets in Parala have not changed much, other aspects of their lives, some closely related to diets, have. Diets seem to be a part of a cultural package, some components of which are more vulnerable to change than others.

\section{Agriculture}\label{sec:agriculture}

Bhils in the Aurangabad region today eat much the same food that their parents, grandparents, and great grandparents used to. They might try city food on their occasional visits to the neighboring towns or at the weekly market at Loni, but their daily meals are still made up of vegetables foraged from the forest and the products from their farms and domestic animals. However, although the species of crops planted and harvested on the farms are much the same as the ones that have been traditionally cultivated, the cultivars of the crops have changed.

In the early 1960s, hybrid varieties of wheat and rice were introduced to India, purportedly to avoid the widespread food shortage that was bound to otherwise occur. These varieties had performed well in Mexico and the U.S., which at this time was one of the largest exporters of wheat. With some tweaking, introduced varieties took off in a big way in India as well. The new crops had significantly higher yields and the surplus production ensured a viable source of income for farmers. Supposedly, with the help of large quantities of wheat imported from the U.S., these varieties saved India from a devastating famine while also ensuring a comfortable livelihood for farmers. The green revolution was widely touted as an unmitigated success. Very soon, new varieties of millets were developed as well.

As is probably apparent from the previous sections (section \ref{sec:lokparyay} in particular), the Bhils did not profit from the beneficence of the green revolution. Many have argued that India's food problems do not arise from a lack of food, but from the lack of purchasing power resulting from social and economic inequities. This certainly seems to have been true for the Bhils. If anything, the green revolution only exacerbated problems that had arisen because of the privatization of land.

Establishment of forest reserves displaced many Bhils. Further rules restricting the harvesting and use of forest products (including non-timber products) forced them to start looking elsewhere for sustenance. Illiteracy and lack of knowledge of how the new world worked left them unable to attain land rights. It also left them unsuited for anything but labor jobs. Many of the growing and quickly industrializing new farms needed cheap labor, as did the sugarcane plantations in Gujarat and other parts of Maharashtra. Desperately requiring some way to feed and clothe themselves, they had no recourse but to turn to labor and servitude.

Even today, despite Lokparyay's efforts, the situation is dire for many of the Bhils. Some still have to work on the sugarcane plantations either because they have no land or because they have no access to water for their crops. Others work on the bigger farms in the area and on the pomegranate plantations that have recently come up. The situation is significantly better for those that have land rights and access to water, but even they constantly face economic pressures that they cannot escape. (Side note: as laborers on the larger farms, their working conditions are far better than they used to be. This is because the Bhils that do own land and have access to water from the Manyad dam do not depend so heavily on manual labor based jobs now. As the number of tribals with land rights increases, the gentry have been forced to increase wages in order to attract the required amount of labor.)

Due to the history of oppression that they have been subjected to and the injustices that often still befall them, all the Bhils that I talked to have come to believe that they need to keep up with the more affluent parts of society. The alternative is the extirpation of their culture and a poverty ridden and miserable existence. Since exclusion from the market economy is not a possibility today, they have had to find ways to earn money and stay relevant.

One of the means through which they achieve this is by growing more cash crops such as cotton. The surplus from their food crops also serves as an important source of income. Since the amount of income is directly proportional to the amount of surplus generated, most of them choose to cultivate high yielding hybrid varieties. However, high yielding varieties are only high yielding under very specific conditions. Higher yields come at the cost of lower drought and pest resistance and higher fertilization requirements. Farmers are thus forced to invest in pesticides and fertilizers, something which they never needed to do when they could depend on the plants' inherent defenses against pests and on manure from their cows for fertilization. The increased dependence on water and the lack of rain also means that the benefits are reaped only by those who have access to irrigation from the Manyad dam.

Additionally, farmers also have to purchase seeds every year since seeds harvested from the hybrid crops are not viable. Often, the money spent on seeds and external inputs outweighs the income from surplus production; which results in the Bhils having to take out loans that are difficult to pay off in the face of sustained drought.

On top of all of this, none of the people I talked to seem to be happy about the new crops. They say that the older varieties not only tasted better, but they also stayed fresh longer after harvesting. They claim that people used to be healthier on the old produce even though they had less of it.

The health claims tie in to larger scale agricultural changes that are occurring concurrently. Although farms in India are not as big and have not been mechanized to the same extent as in the US, there does seem to be a trend in this direction. Big farms in India are only about 120 acres or less, and are probably even smaller in the Parala region (need to find out what the figures are). In many parts of the country, they have been buying up and replacing smaller farms (mostly in the 10 - 30 acre range; farms smaller than that can be managed by a family and the larger ones can invest in machines and infrastructure) and have come to rely more heavily on technological solutions.

In Parala as well, the big farms have been getting bigger and have been investing more heavily in machinery and in external inputs. However, spraying of pesticides, cotton picking, and fertilizer application is still done by hand or with the help of simple, manually operated machines. Hired laborers, which include the people that I talked to, do all of this work and are resultantly exposed to the various chemicals used. I do not yet know what synthesized chemicals are used and whether they have been proven to be harmful to human health, but my interviewees firmly believe that the number of health complaints among the laborers has increased. Women are employed for picking cotton and they spend long hours in the fields handling pesticide treated plants. They often have their children with them and it seems as though women's and children's health has deteriorated more than that of the tribal men.

The big farms almost exclusively grow only those crops that can be cultivated in high densities as monocultures or that can generate high returns. Thus legumes are excluded from these farms and only cotton and cereals such as corn, wheat, and millets are grown. Aggressive weeding ensures that no other plants can survive on these farms.

This is a far cry from the tribal farms where multiple food crops are cultivated, often in alternating rows or mixed in with each other. Cereals are mostly grown as monocultures, but shade tolerant legumes such as \textit{harbhare} can be and often are mixed in. Weeds are also allowed to grow if they are recognized to be edible or medicinal plants. These traditions are being slowly eroded by the incursion of hybrid cereal varieties, but will likely be kept alive by the Bhils' heavy dependence on legumes and beans in their diets.

Surveys undertaken by Bapu show that not only is there a greater variety of cultivated plants in the Bhils' plots, but that the diversity and abundance of wild plants is higher in these plots as well. The larger farms have a few well known tree species growing along the edges and almost no vines and herbs, while the corresponding numbers for the Bhil farms are often an order of magnitude higher. If patterns of land use change seen elsewhere in India (larger farms replacing small ones) were to occur here (quite likely; more on this in the next section), biodiversity, at least in terms of species diversity, will suffer.

There is a somewhat related change that is occurring in urban India that is counter-acting these pressures to move towards hybrid crops and bigger, more mechanized farms. The economic upper and middle classes of India are becoming more conscious of what they eat. There is now a greater demand for native (\textit{gavran}) food varieties and for food that is perceived to be more rural. The rustic \textit{bhakri} made of millets is becoming increasingly popular and is more easily available. Lokparyay is using these social movements to create a market for native food varieties in Parala. They hope that this could stem the proliferation of seemingly unsustainable and probably less healthy crop varieties and farming methods while also providing the tribal people with a means of livelihood that supports their traditions. It might be that these movements are transient and even if they are not, that they might not percolate to other parts of society. If changes in food consumption patterns, determined in large part by business interests, remain localized to a small part of society, it seems highly unlikely that they could alter the current course of events.

\newpage

\subsection{Material, educational, and other changes}

There are other changes that are taking place here as well that I see as a response to economic and social pressures. Many of these are ones that are occurring throughout India.

One instance of such a change is an increasing consumer mentality. People all over India seem to want more and better things, regardless of the utility that they derive from them. Phones, vehicles, household goods, all seem to be in constant need of upgrades and replacements. This seems to have become true in Parala as well. Mobile phones, as in the rest of India, have become nearly ubiquitous. They have also become almost essential for functioning in the Indian economy and society. However, like many other Indians, a number of the Bhils have multiple phones which they replace frequently. The Thakars are apparently the early innovators who bring in new technology and customs to the village, but the Bhils adopt these novelties quickly.

Once the basic needs of land and a reliable source of water are met, increasing amounts of money are spent on convenience and ostentation. This is not evident among older generations, but the newer generations are quick to invest in fashion trends or phones. There are of course technological innovations such as tractors or electricity that make life easier for the Bhils, but the same cannot be said for constant upgrades that seem to deliver only marginal benefits.

The primary reason for this development seems to be the desire to keep up with the neighbors. As one of my interviewees explained it to me, once a person sees a peer wearing trendy clothes and being noticed (positively) for doing so, he does not want to be seen as slovenly or backward. He is tempted to follow suit and if finances permit, he will. It appears that improved financial circumstances have been necessary to initate this change, but that social perceptions are necessary to propagate it. In an environment of increased exposure to media, advertising and marketing strongly influence what is considered fashionable or even necessary.

I do not know whether the perceived need for keeping up with neighbors itself is a new development or whether this is a universal human trait. It is hard to imagine a society in which individuals are not constantly comparing themselves to their peers, but it is also very hard for me to imagine how this trait would play out in a forest dwelling society that has limited access to material goods. Membership to any group seems to impose certain standards that need to be adhered to. These standards themselves might change as conditions change, but there might well be simple rules to explain what these standards are and how they change in response to changing environments.

As the number of people that the Bhils are forced to interact with increases, they have also had to learn to communicate with them. While most Bhils today still speak the Bhil language, many of them can also speak Marathi fluently. Local schools only teach in Marathi; so growing literacy promotes this trend.

Another consequence of growing literacy could well be emigration to cities. Very few of the Parala Bhils have thought about moving to cities, mostly because they lack the education required to obtain jobs there. Even those who have tried living in places like Mumbai have had to return eventually because they have found it impossible to find housing and adequate means to support their families there. With the Bhil children now able to avail of better education at well funded boarding schools, that will most probably change. It seems highly unlikely that after realizing the possibility of drastic economic advancement unlocked by education at reputed schools, children would want to revert to farming or that their parents would want them to do so.

This mode of urbanization would be different from the more common model whereby dispossessed and destitute villagers are forced to move to cities in search of means to sustain themselves. In the latter case, immigrants inevitably end up in slums and struggle to make ends meet. The educated Bhil would probably be much better off.

In terms of land use as well, the effects could turn out to be very different. The emigration of villagers in desperate need of economic opportunities is accompanied by land retirement or by the incorporation of abandoned land into larger farms. The effects of land retirement/abandonment are not well known. It also does 


\end{document}