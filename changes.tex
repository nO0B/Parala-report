\documentclass[report.tex]{subfiles}

\begin{document}

\chapter{Changes}\label{chp:changes}

The Bhils have witnessed the rise and fall of numerous cultures and a few illustrious civilizations over the many centuries of their existence (subsection \ref{subsec:history} in the previous chapter provides a very brief history of the Bhil people). Some experiences with other cultures have been affable, while others have been markedly hostile. Bhil traditions have likely changed repeatedly in response to these encounters (probably more in response to the hostile ones).

Their most recent major run-in has been with a culture founded on capitalism and individual rights. Exposure to this culture probably began in the British colonial era with the privatization of the forests that used to be their homes. Exposure has continued post independence as free India has continued along a path of development that it was introduced to by the British. This path is decidedly anti-rural and has forced the Bhils to adapt by changing many aspects of their lives. This chapter deals with what I think some of these changes are. Similar trends are being observed all over India but to my knowledge, not much has been done to understand the social and economic factors driving them. Therefore, in this chapter I will also attempt to identify the forces that I think are important. Although people's diets in Parala have not changed much, other aspects of their lives, some closely related to diets, have. Diets seem to be a part of a cultural package, some components of which are more vulnerable to change than others.

\section{Agriculture}\label{sec:agriculture}

Bhils in the Aurangabad region today eat much the same food that their parents, grandparents, and great grandparents used to. They might try city food on their occasional visits to the neighboring towns or at the weekly market at Loni, but their daily meals are still made up of vegetables foraged from the forest and the products from their farms and domestic animals. However, although the species of crops planted and harvested on the farms are much the same as the ones that have been traditionally cultivated, the cultivars of the crops have changed.

In the early 1960s, hybrid varieties of wheat and rice were introduced to India, purportedly to avoid the widespread food shortage that was bound to otherwise occur. These varieties had performed well in Mexico and the U.S., which at this time was one of the largest exporters of wheat. With some tweaking, introduced varieties took off in a big way in India as well. The new crops had significantly higher yields and the surplus production ensured a viable source of income for farmers. Supposedly, with the help of large quantities of wheat imported from the U.S., these varieties saved India from a devastating famine while also ensuring a comfortable livelihood for farmers. The green revolution was widely touted as an unmitigated success. Very soon, hybrid varieties of millets were introduced as well.

As is probably apparent from the previous sections (section \ref{sec:lokparyay} in particular), the Bhils did not profit from the beneficence of the green revolution. Many have argued that India's food problems do not arise from a lack of food, but from the lack of purchasing power resulting from social and economic inequities. This certainly seems to have been true for the Bhils. If anything, the green revolution only exacerbated problems that had arisen from the privatization of land. Establishment of forest reserves had already displaced many Bhils. Strict rules governing the harvesting and use of forest products (including non-timber products) had forced them to start looking elsewhere for sustenance. Illiteracy and lack of knowledge of how the new world worked left them unsuited for anything but labor jobs. Many of the growing and quickly industrializing new farms needed cheap labor, as did the sugarcane plantations in Gujarat and other parts of Maharashtra. Desperately requiring some way to feed and clothe themselves, they had no recourse but to turn to become laborers.

Even today, despite Lokparyay's efforts, the situation is still dire for many of the Bhils. Some still have to work on the sugarcane plantations either because they have no land or because they have no access to water for their crops. However, as laborers on the larger farms, their working conditions are far better than they used to be. This is because the Bhils that do own land and have access to water from the Manyad dam do not depend so heavily on manual labor based jobs now. Hence, the gentry have been forced to increase wages in order to attract the required amount of labor.

Due to the history of oppression that they have been subjected to and the injustices that often still befall them, all the Bhils that I talked to have come to believe that they need to keep up with the more affluent parts of society. The alternative is the extirpation of their culture and a poverty ridden and miserable existence. Since exclusion from the market economy is not a possibility today, they have had to find ways to earn money and stay relevant.

One of the means through which they achieve this is by growing more cash crops such as cotton. The surplus from their food crops also serves as an important source of income. Since the amount of income is directly proportional to the amount of surplus generated, most of them choose to cultivate high yielding hybrid varieties. However, high yielding varieties are only high yielding under very specific conditions. Higher yields come at the cost of lowered drought resistance, decreased pest resistance, and the need for higher amounts of fertilizer. This requires farmers to invest in pesticides and fertilizers, something which they never needed when they could depend on the plants' inherent defenses against pests and on manure from their cows for fertilization. The increased dependence on water and the lack of rain also means that the benefits are reaped only by those who have access to irrigation from the Manyad dam.

Farmers also have to invest in seeds every year since seeds from the new varieties are not viable. Often, the money spent on seeds and external inputs outweighs the income from surplus production; which results in the Bhils having to take out loans.




There are larger scale agricultural shifts afoot as well. The bigger farms have all become businesses where the owner does not depend on the produce for sustenance at all. In fact, the owner is usually a city dweller who only occasionally visits his farms. These farms have become monocultures of corn, bajra, or wheat. A number of pomegranate plantations have sprung up in the area as well.

These shifts in agricultural patterns seem to be a response to urban demands. Diets in urban India are shifting towards 

\end{document}