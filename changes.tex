\documentclass[report.tex]{subfiles}

\begin{document}

\chapter{Changes}

The Bhils have witnessed the rise and fall of numerous cultures over the many centuries of their existence (subsection \ref{subsec:history} in the previous chapter provides a very brief history of the Bhil people). Some experiences with other cultures have been affable, while others have been markedly hostile. Bhil traditions have likely changed repeatedly in response to these encounters (probably more in response to the hostile ones).

Their most recent major run-in has been with a culture founded on capitalism. Exposure to this culture probably began in the British colonial era with the privatization of the forests that used to be their home. Exposure has continued post independence because `free' India has continued along a path of development that was introduced to it by the British. This path is decidedly anti-rural and has forced the Bhils to adapt by changing many aspects of their lives. This chapter deals with what I think these changes are. Many of these changes are being observed all over India but to my knowledge, not much has been done to understand the social and economic factors driving these changes. In this chapter, I will attempt to identify the forces that I think are important. Although people's diets here have not changed much, it seems to me that the forces that are driving changes in other aspects of people's lives might also be responsible for dietary shifts.

\section{Lifestyles}\label{sec:lifestyles}

\end{document}