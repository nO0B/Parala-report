\documentclass[report.tex]{subfiles}

\begin{document}

\chapter{Changes}\label{chp:changes}

The Bhils have witnessed the rise and fall of numerous cultures and a few illustrious civilizations over the many centuries of their existence (subsection \ref{subsec:history} in the previous chapter provides a very brief history of the Bhil people). Some experiences with other cultures have been affable, while others have been markedly hostile. Bhil traditions have likely changed repeatedly in response to these encounters (probably more in response to the hostile ones).

Their most recent major run-in has been with a culture founded on capitalism and individual rights. Exposure to this culture probably began in the British colonial era with the privatization of the forests that used to be their homes. Exposure has continued post independence as free India has continued along a path of development that it was introduced to by the British. This path is decidedly anti-rural and has forced the Bhils to adapt by changing many aspects of their lives. This chapter deals with what I think some of these changes are. Similar trends are being observed all over India but to my knowledge, not much has been done to understand the social and economic factors driving them. Therefore, in this chapter I will also attempt to identify the forces that I think are important. Although people's diets in Parala have not changed much, other aspects of their lives, some closely related to diets, have. Diets seem to be a part of a cultural package, some components of which are more vulnerable to change than others.

\section{Agriculture}\label{sec:agriculture}

Bhils in the Aurangabad region today eat much the same food that their parents, grandparents, and great grandparents used to. They might try city food on their occasional visits to the neighboring towns or to the weekly market at Loni, but their daily meals are still made up of vegetables foraged from the forest and the products from their farms and domestic animals. However, although the species of crops planted and harvested on the farms are also much the same as the ones that have been traditionally cultivated, the cultivars of the crops have changed.

In the early 1960s, hybrid varieties of wheat and rice were introduced to India, purportedly to avoid the widespread food shortage that were bound to otherwise occur. These varieties had performed well in Mexico and the U.S., which at this time was one of the largest exporters of wheat. With some tweaking, introduced varieties took off in a big way in India as well. The new crops had significantly higher yields and the surplus production ensured a viable source of income for farmers. Supposedly, with the help of large quantities of wheat imported from the U.S., these varieties saved India from a devastating famine.

Needless to say, the Bhils did not particularly benefit from this revolution.

\end{document}