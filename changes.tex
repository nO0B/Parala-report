\documentclass[report.tex]{subfiles}

\begin{document}

\chapter{Changes}

The Bhils have witnessed the rise and fall of numerous cultures and a few illustrious civilizations over the many centuries of their existence (subsection \ref{subsec:history} in the previous chapter provides a very brief history of the Bhil people). Some experiences with other cultures have been affable, while others have been markedly hostile. Bhil traditions have likely changed repeatedly in response to these encounters (probably more in response to the hostile ones).

Their most recent major run-in has been with a culture founded on capitalism and individual rights. Exposure to this culture probably began in the British colonial era with the privatization of the forests that used to be their homes. Exposure has continued post independence as free India has continued along a path of development that it was introduced to by the British. This path is decidedly anti-rural and has forced the Bhils to adapt by changing many aspects of their lives. This chapter deals with what I think some of these changes are. Similar trends are being observed all over India but to my knowledge, not much has been done to understand the social and economic factors driving them. Therefore, in this chapter I will also attempt to identify the forces that I think are important. Although people's diets in Parala have not changed much, it seems to me that the forces that are driving changes in other aspects of people's lives might also be responsible for dietary shifts. Diets seem to be a part of a cultural package, some components of which are more vulnerable to change than others.

\section{Lifestyles}\label{sec:lifestyles}

The establishment of forest reserves and private property rights has caused major lifestyle changes for many peoples, including the Bhils.

\end{document}