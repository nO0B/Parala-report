\documentclass[report.tex]{subfiles}

\begin{document}

\chapter{Changes}

The Bhils have witnessed the rise and fall of numerous cultures over the many centuries of their existence (subsection \ref{subsec:history} in the previous chapter provides a very brief history of the Bhil people). Some experiences with other cultures have been affable, while others have been markedly hostile. Bhil traditions have likely changed repeatedly in response to these encounters (probably more in response to the hostile ones).

Their most recent major run-in has been with a culture founded on capitalism. Exposure to this culture probably began in the British colonial era with the privatization of the forests that used to be their home. Exposure has continued post independence as free India has continued along a path of development that was introduced to it by the British. This path is decidedly anti-rural and has forced the Bhils to adapt by changing many aspects of their lives. This chapter deals with what I think these changes are. Similar trends are being observed all over India but to my knowledge, not much has been done to understand the social and economic factors driving them. In this chapter, I will also attempt to identify the forces that I think are important. Although people's diets in Parala have not changed much, it seems to me that the forces that are driving changes in other aspects of people's lives might also be responsible for dietary shifts. Diets are a part of a cultural package, some components of which are more vulnerable to change than others.

\section{Lifestyles}\label{sec:lifestyles}

The biggest change that has befallen the Bhils is in the way they live their lives. As mentioned earlier (subsection \ref{subsec:history}), the Bhils are traditionally a forest-dwelling tribe that used to practice a form of shifting agriculture. In the early part of the 1800s, a few Bhils in different parts of the country (Mewar and west Khandesh, for example) had been organized into corps by some of the more enterprising and generous British officers. Others worked as night watchmen in some villages in Maharashtra, Gujarat, and Rajasthan. Not being cognizant of the precise history of the Bhils of the Aurangabad region, I have no idea how much they were influenced by British rule. However, there are some larger scale forces that were in action that I think must have introduced tribals of this region, including the Bhils, to the ideas and philosophies of the new ruling class.

In 1865, the British laid claim to most of India's forests through the Forest Act. Extraction of any resources from forest reserves without official permission was prohibited. However, the size of the country and the extent of India's forests made this ordinance difficult to enforce. Although persecuted, Bhils in other parts of the country were still able to live in and off the forests. The situation was exacerbated by the severe droughts and famines in the 1890s and the first decade of the 1900s. Shortage of food and water caused many of the Bhils to move out of the forests and look for sustenance elsewhere. Many starved to death. I am not sure if this had any lasting consequences in the Aurangabad region, but this was also the time when India's railway and road network had started reaching remote areas of the country. Timber and farmland had already started taking a toll on Maharashtra's forests and the growing railways and roads speeded up this process. During this time, some commercial exchange between the Bhils of Aurangabad and urban populations had likely started occurring. Commercial exchanges are inevitably accompanied by corresponding cultural exchanges.

Cultural exchange brought with it notions of private land ownership, an idea that the Bhils had probably already encountered because of the Forest Act of 1865. Probably the biggest change that private ownership of land has brought with it is the sedentarization of habitually mobile populations. Bhils used to live off the produce derived from cleared land for as long as the soil was fertile.

\end{document}